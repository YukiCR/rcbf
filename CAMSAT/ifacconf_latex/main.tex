%===============================================================================
% ifacconf.tex 2022-02-11 jpuente  
% 2022-11-11 jpuente change length of abstract
% Template for IFAC meeting papers
% Copyright (c) 2022 International Federation of Automatic Control
%===============================================================================
\documentclass{ifacconf}

\usepackage{graphicx}      % include this line if your document contains figures
\usepackage{subfig}
\usepackage{natbib}        % required for bibliography
\usepackage{amsmath}
\usepackage{amssymb}
\usepackage{enumitem}
\usepackage{balance} % better balance in practice

% --- use the following code to use hyperref with ifacconf ---
% see also: https://tex.stackexchange.com/questions/393690/illegal-unit-of-measure-error-when-using-hyperref-in-the-ifacconf-class
\makeatletter
\let\old@ssect\@ssect % Store how ifacconf defines \@ssect
\makeatother

\usepackage{hyperref}

\makeatletter
\def\@ssect#1#2#3#4#5#6{%
  \NR@gettitle{#6}% Insert key \nameref title grab
  \old@ssect{#1}{#2}{#3}{#4}{#5}{#6}% Restore ifacconf's \@ssect
}
\makeatother
% --- end of hyperref ---

\newtheorem{definition}{Definition}
\newtheorem{asmp}{Theorem}
\newtheorem{assumption}[asmp]{Assumption}

\DeclareMathOperator*{\argmin}{arg\,min}
%===============================================================================
\begin{document}
\begin{frontmatter}

\title{Integrating High-Level Decisions with Distributed Safey Filter for Multi-Satellite Collision Avoidance} 
% Title, preferably not more than 10 words.

\thanks[footnoteinfo]{Sponsor and financial support acknowledgment
goes here. Paper titles should be written in uppercase and lowercase
letters, not all uppercase.}

\author[First]{First A. Author} 
\author[Second]{Second B. Author, Jr.} 
\author[Third]{Third C. Author}

\address[First]{National Institute of Standards and Technology, 
   Boulder, CO 80305 USA (e-mail: author@ boulder.nist.gov).}
\address[Second]{Colorado State University, 
   Fort Collins, CO 80523 USA (e-mail: author@lamar. colostate.edu)}
\address[Third]{Electrical Engineering Department, 
   Seoul National University, Seoul, Korea, (e-mail: author@snu.ac.kr)}

\begin{abstract}                % Abstract of 50--100 words
   % Miniaturization and clustering exacerbate collision risk of future satellites.
   % In this paper, we propose a distributed framework for inter-satellite collision avoidance with tunable priority.
   % We first deduce the collision-free condition of multi-satellite systems based on High Order Control Barrier Function techniques. 
   % We then decouple such a condition for respective satellites, and further encode safety on the nominal controller in the form of distributed safety filter.
   % By introducing priority parameter in the safety filter, the responsibility of evading collisions between satellite pairs becomes tunable, making it possible to modify the swarm behavior.
   % Based on such a mechanism, we further showcase the safety filter's ability to cooperate with high level decisions: cooperating with optimization to approximate global optimal behavior and cooperating with Large Language Models to accommodate to tasks, respectively.
   % Theoretical analysis have proved the safety guarantee of the safety filter and numerical Experiments validated the effectiveness of proposed method.

   Satellite miniaturization and dense constellation deployments exacerbate collision risks in future orbital operations. 
   While numerous collision avoidance strategies have been proposed, few reconcile agent-level safety with mission-level efficiency.
   In this paper, we propose a distributed inter-satellite collision avoidance framework embeded with high-level tuned priorities.
   First, we formulate ``safe protocol'' constraints between satellite pairs and enforce these constraints on their nominal controllers through distributed safety filters, which establishs collision-free coordination of the whole swarm.
   By introducing tunable priority parameters within the safety filter, collision evasion responsibilities become dynamically adjustable, enabling swarm behavior adaptation. 
   We further demonstrate two methods to integrate with high-level decisions: cooperating with optimization to approximate global reference behaviors and cooperating with Large Language Models to accommodate to tasks, respectively.
   Theoretical analysis proves the safety guarantees, while numerical experiments validate the framework's efficacy.
\end{abstract}

\begin{keyword}
Multi-satellite, Collision Avoidance, Control Barrier Function
\end{keyword}

\end{frontmatter}
%===============================================================================

\section{Introduction}
This document is a template for \LaTeXe. If you are reading a paper or
PDF version of this document, please download the electronic file
\texttt{ifacconf.tex}. You will also need the class file
\texttt{ifacconf.cls}. Both files are available on the IFAC web site.

Please stick to the format defined by the \texttt{ifacconf} class, and
do not change the margins or the general layout of the paper. It
is especially important that you do not put any running header/footer
or page number in the submitted paper.\footnote{
This is the default for the provided class file.}
Use \emph{italics} for emphasis; do not underline.

Page limits may vary from conference to conference. Please observe the 
page limits of the event for which your paper is intended. \cite{Abl:56}

\section{Preliminaries}
\par We review High Order Control Barrier Function here, which is the fundamental technique in this paper.

Consider a general continuous time control-affine system
\begin{equation} \label{eqn:system}
  \dot{\boldsymbol{x}} = \boldsymbol{f}(\boldsymbol{x}) + \boldsymbol{g}(\boldsymbol{x}) \boldsymbol{u} ,
\end{equation}
where $x \in \mathcal{X} \subset \mathbb{R}^n$ is the state and $u \in \mathcal{U} \subset \mathbb{R}^m$ is the system input. $\boldsymbol{f}$ and $\boldsymbol{g}$ are locally Lipschitz continuous functions. The high order control barrier function is defined as follows.

\begin{definition}\label{def:HOCBF}
   Given a system (\ref{eqn:system}) with relative degree $r_b$ and $r_b$-th order differentiable function $h(\boldsymbol{x})$, define a series of functions $\Psi_r, r = 0, \dots, r_b$ recursively as
   \begin{equation} \label{eqn:cbfRecursive}
     \begin{aligned}
       \Psi_0 &= h(\boldsymbol{x}), \\
       \Psi_k &= \dot{\Psi}_{k-1} + \alpha_k\left( \Psi_{k-1}(\boldsymbol{x}) \right), k = 1,\dots,r_b,
     \end{aligned}
   \end{equation}
   where $\alpha_k(\cdot)$ are extended class $\mathcal{K}_{\infty}$ functions \footnotemark. 
   
   \par The zero-superlevel set of these defined functions are
   \footnotetext{A continuous function $\alpha$ : $\mathbb{R} \to \mathbb{R}$ is an extended class $\mathcal{K}_{\infty}$ function if $\alpha(0) = 0$ and $\lim_{x\to \pm \infty}(x) = \pm \infty$.}
   \begin{equation}
     \mathfrak{S}_r = \left\{ \boldsymbol{x} \in \mathbb{R}^n \mid \Psi_r(\boldsymbol{x}) \ge 0 \right\}, r = 0,\dots,r_b.
   \end{equation}
   $h$ is a \textit{High Order Control Barrier Function (HOCBF)} for system (\ref{eqn:system}), if there exists extended class $\mathcal{K}_{\infty}$ functions $\alpha_1, \dots, \alpha_{r_b}$ such that
   \begin{equation} \label{eqn:cbfConstraint}
     \Psi_{r_b}(\boldsymbol{x}) \ge 0
   \end{equation}
   stands for any $(\boldsymbol{x},t) \in \mathfrak{S} \times [0,\infty]$, where $\mathfrak{S} = \bigcap_{r = 0}^{r_b} \mathfrak{S}_r$. 

\end{definition}

\begin{thm}[\cite{HOCBFxiaowei}] \label{thm:HOCBF}
   \par Following the definitions in Definition \ref{def:HOCBF}, once $h$ is a HOCBF for system (\ref{eqn:system}), $\mathfrak{S}$ would be a \textit{forward invariant set} for the system, i.e., $\boldsymbol{x}(0) \in \mathfrak{S}, \boldsymbol{x}(t) \in \mathfrak{S}, \forall t > 0$.
\end{thm}

\section{Problem Formulation}
\subsection{System Modelling}
\par We consider $N$ satellite agents whose dynamics governed by Clohessy-Wiltshire equantions \cite[]{CWequations}.
The dynamics of agent $i$ in the reference orbit frame is described as
\begin{equation} \label{eqn:dynamics1}
   \dot{\boldsymbol{x}}_i = 
      \begin{bmatrix}
         \dot{\boldsymbol{p}}_i \\ \dot{\boldsymbol{v}}_i
      \end{bmatrix} = 
      \begin{bmatrix}
         \boldsymbol{v}_i \\ \boldsymbol{f}_{vi}
      \end{bmatrix} + 
      \begin{bmatrix}
         \boldsymbol{0} \\ E
      \end{bmatrix} \boldsymbol{u}_i,
\end{equation}
and
\begin{equation} \label{eqn:dynamics2}
   \boldsymbol{f}_{vi} = 
      \begin{bmatrix}
         -2 \omega v_{yi} \\
         2\omega v_{xi} + 3\omega^2 v_{yi} \\
         \omega^2 p_{zi}
      \end{bmatrix},
\end{equation}
where $\boldsymbol{p}_i = [p_{xi}~p_{yi}~p_{zi}]^{\top} \in \mathbb{R}^3, \boldsymbol{v} = [v_{xi}~v_{yi}~v_{zi}]^\top \in \mathbb{R}^3$ and $\boldsymbol{u}_i \in \mathbb{R}^3$ are the position, velocity and acceleration of agent $i$, respectively.
$\boldsymbol{0}$ and $E$ are zero matrix and identity matrix with proper size, and $\omega \in \mathbb{R}$ is the angular velocity of the reference orbit.

\begin{assumption}
$\boldsymbol{x}_j, j = 1, \dots, N$ and $\omega$ are known for any agent $i, i = 1, \dots, N$ and the high-level decision module.
\end{assumption}
\par This assumption is justified since agent $i$ could estimate the state of agent $j$ through relative position estimation techniques \cite[]{XXX}, and the high-level decision module (possibly ground control station or space station) could get these information through observation \cite[]{XXX} or communication.

\subsection{Safety Requirement}
\par The safety requirement of satellite agents is to keep the safety distance between from each other. 
Denote $r_i \in \mathbb{R}^+$ to be the safety distance of agent $i$, the safety requirement between agent $i$ and agent $j$ is then keeping the set
\begin{equation}
   \mathfrak{S}_{0ij} = \{\boldsymbol{x}_i, \boldsymbol{x}_j \mid d_{ij} = \| \boldsymbol{p}_i - \boldsymbol{p}_j \| \ge R_{ij} = r_i + r_j \}
\end{equation}
forward invariant. 
And the safety requirement of the swarm is to keep 
\begin{equation}
   \mathfrak{S}_0 = \bigcap_{i \neq j}\mathfrak{S}_{0ij}, ~i,j = 1, \dots, N
\end{equation}
forward invariant.

\subsection{Main Objective}
\par The main objective of this paper is twofold:
\begin{enumerate}[label=\arabic*)]
   \item Ensuring agent-level safety: for each agent $i$, given the local reference control $\boldsymbol{u}_{ri}$ and observation $\boldsymbol{X} = [\boldsymbol{x}_1^{\top}~\dots~\boldsymbol{x}_n^{\top}]^{\top}$, synthesis the safeguarding policy $\boldsymbol{u}_i = \boldsymbol{\pi}_{i}(\boldsymbol{u}_{ri}, \boldsymbol{X})$ to keep $\mathfrak{S}_0$ forward invariant for the swarm through control $\boldsymbol{U} = [\boldsymbol{u}_1^\top~\dots~\boldsymbol{u}_N^\top]^\top$.
   \item Cooperating with high-level decisions: 
      \begin{itemize}
         \item Given the global reference control $\boldsymbol{U}_r = [\boldsymbol{u}_{gr1}^{\top}~\dots~\boldsymbol{u}_{grn}]^\top$, tune $\boldsymbol{\pi}_i, i = 1, \dots, N$ to approximate $\boldsymbol{U}_r$ with $\boldsymbol{U}$.
         \item Given the mission discribed with nature language, tune $\boldsymbol{\pi}_i, i = 1, \dots, N$ to adjust the collision evasion responsibily of agent $i$ based on its mission-level importance. 
      \end{itemize}
\end{enumerate}

\section{Distributed Safety Filter Design}
\par We first introduce the ``safe protocol'' constraint formulation, then give the distributed safety filter design based on the ``safe protocol''. 

\par For satellite $i$ and satellite $j$, the ``safe protocol'' set for satellite $i$ corresponding to satellite $j$ is a half-space described as
\begin{equation}
   \begin{aligned}
      \mathcal{S}_{ij} = \left\{ \boldsymbol{u}_i \mid -\hat{\boldsymbol{n}}_{ij}^\top \boldsymbol{u}_i \le \vphantom{\frac{1}{d_{ij}}} \right. 
      & \hat{\boldsymbol{n}}_{ij}^{\top} \left[(\alpha_1 + \alpha_2)\boldsymbol{v}_i + \boldsymbol{f}_{vi} \right] \\
      &+ p_{ij} \left[ \alpha_1 \alpha_2 (d_{ij} - R_{ij}) \vphantom{\frac{1}{d_{ij}}} \right. \\
      & \left. \left. + \frac{1}{d_{ij}}\left( \|\boldsymbol{v}_{ij}\|^2 - ( \hat{\boldsymbol{n}}_{ij}^{\top}\boldsymbol{v}_{ij} )^2 \right) \right] \right\},
   \end{aligned}
\end{equation}
where $\hat{\boldsymbol{n}}_{ij} = (\boldsymbol{p}_i - \boldsymbol{p}_j)/\| \boldsymbol{p}_i - \boldsymbol{p}_j \|$ is the unit vector pointing from agent $j$ to agent $i$, 
$\alpha_1, \alpha_2 \in \mathbb{R}^{+}$ are positive parameters and 
$\boldsymbol{v}_{ij} = \boldsymbol{v}_i - \boldsymbol{v}_j$ is the relative velocity between agent $i$ and agent $j$.
$p_{ij} \in \mathbb{R}$ is the priority parameter, reresenting agent $i$'s priority over agent $j$. 

\par The following result renders that two satellites are collision-free if both of the them follow the ``safety protocol''.

\begin{thm}
   For systems discribed as (\ref{eqn:dynamics1}) and (\ref{eqn:dynamics2}), $\mathfrak{S}_{0ij}$ is forward invariant if $\boldsymbol{u}_i \in \mathcal{S}_{ij}, \boldsymbol{u}_j \in \mathcal{S}_{ji}$ and $p_{ij} + p_{ji} \le 1$.
\end{thm}
\begin{pf}
The proof mainly leverages Theorem \ref{thm:HOCBF}.
Let
\begin{equation}
    h_{ij} = d_{ij} - R_{ij} = \Psi_{0ij}
\end{equation}
be the HOCBF candidate to keep agent $i$ and agent $j$ collision-free.
By choosing positive proportional functions as class $\mathcal{K}_{\infty}$ functions, it follows the definition that
\begin{equation}
\begin{aligned}
      \Psi_{1ij} =& \dot{\Psi}_{0ij} + \alpha_1 (\Psi_{0ij}) \\
                 =& \frac{\boldsymbol{p}_i^{\top} - \boldsymbol{p}_j^\top}{d_{ij}} (\boldsymbol{v}_1 - \boldsymbol{v}_2) + \alpha_1 (d_{ij} - R_{ij}) \\
                 =&  \hat{\boldsymbol{n}}_{ij}^\top \boldsymbol{v}_{ij} + \alpha_1 \cdot \Psi_{0ij}.
\end{aligned}
\end{equation}
To get the constraint on control, we further define
\begin{equation}
   \begin{aligned}
      \Psi_{2ij} =& \dot{\Psi}_{1ij} + \alpha_2 \Psi_{1ij} \\
                 =& \frac{1}{d_{ij}^2}\left( (\boldsymbol{v}_1 - \boldsymbol{v}_2)^{\top}d_{ij} - (\hat{\boldsymbol{n}}_{ij}^\top \boldsymbol{v}_{ij})(\boldsymbol{p}_1 - \boldsymbol{p}_2)^{\top} \right) \boldsymbol{v}_{ij} \\
                 &+ \hat{\boldsymbol{n}}_{ij}^\top (\boldsymbol{u}_1 + \boldsymbol{f}_{v1} - \boldsymbol{u}_2 - \boldsymbol{f}_{vj}) + \alpha_1 \hat{\boldsymbol{n}}_{ij}^\top \boldsymbol{v}_{ij} \\
                 &+ \alpha_2 (\hat{\boldsymbol{n}}_{ij}^\top \boldsymbol{v}_{ij} + \alpha_1 \Psi_{0ij}) \\
                 =& \hat{\boldsymbol{n}}_{ij}^\top \left(\boldsymbol{u}_1 - \boldsymbol{u}_2 + (\alpha_1 + \alpha_2)\boldsymbol{v}_{ij} + \boldsymbol{f}_{v1} - \boldsymbol{f}_{v2} \right) \\
                 &+ \frac{1}{d_{ij}} \left( \|\boldsymbol{v}_{ij}\|^2 -  (\hat{\boldsymbol{n}}_{ij}^{\top}\boldsymbol{v}_{ij} )^2 \right) + \alpha_1 \alpha_2 (d_{ij} - R_{ij})
   \end{aligned}
\end{equation}
Given that $\boldsymbol{u}_i \in \mathcal{S}_{ij}$ and $\boldsymbol{u}_j \in \mathcal{S}_{ji}$, by adding up the inequalities in the definition of $\mathcal{S}_{ij}$ and $\mathcal{S}_{ji}$ and substituting $\hat{\boldsymbol{n}}_{ji} = -\hat{\boldsymbol{n}}_{ij}$ into the inequality, one can get
\begin{equation}
   \begin{aligned}
      \tilde{\Psi}_{2ij} =& \hat{\boldsymbol{n}}_{ij}^\top \left(\boldsymbol{u}_1 - \boldsymbol{u}_2 + (\alpha_1 + \alpha_2)\boldsymbol{v}_{ij} + \boldsymbol{f}_{v1} - \boldsymbol{f}_{v2} \right) \\
      &+ (p_{ij} + p_{ji})\left[ \frac{1}{d_{ij}} \left( \|\boldsymbol{v}_{ij}\|^2 -  (\hat{\boldsymbol{n}}_{ij}^{\top}\boldsymbol{v}_{ij} )^2 \right) \right. \\
      &+ \left. \alpha_1 \alpha_2 (d_{ij} - R_{ij}) 
            \vphantom{ \frac{1}{d_{ij}} \left( \|\boldsymbol{v}_{ij}\|^2 -  (\hat{\boldsymbol{n}}_{ij}^{\top}\boldsymbol{v}_{ij} )^2 \right) } 
         \right] \ge 0
   \end{aligned}
\end{equation}
Notice that $\|\boldsymbol{v}_{ij}\|^2 -  (\hat{\boldsymbol{n}}_{ij}^{\top}\boldsymbol{v}_{ij} )^2  \ge 0$ and $d_{ij} - R_{ij} \ge 0$ stands for $(\boldsymbol{x}_i, \boldsymbol{x}_j) \in \mathfrak{S}_{0ij}$, $\Psi_{2ij} \ge \tilde{\Psi}_{2ij} \ge 0$ then stands for any $p_{ij}$ and $p_{ji}$ satisfying $p_{ij} + p_{ji} \le 1$.
Consequently, according to Definition \ref{def:HOCBF} and Theorem \ref{thm:HOCBF}, $h_{ij}$ is a valid HOCBF and $\mathfrak{S}_{0ij}$ is then a forward invariant set for agent $i$ and agent $j$.
\end{pf}

Based on the properties of ``safe protocol'' set, we further design the distributed safeguarding policy in a minimum invasive way as
\begin{equation}
   \boldsymbol{\pi}_i(\boldsymbol{u}_{ri}, \boldsymbol{X}) = 
   \left\{
   \begin{aligned}
      &\mathop{\arg \min}_{\boldsymbol{u} \in \cap_{j \neq i} \mathcal{S}_{ij}} \| \boldsymbol{u} - \boldsymbol{u}_{ri} \|^2,    &\cap_{j \neq i} \mathcal{S}_{ij} \neq \varnothing, \\
      &\mathop{\arg \min}_{\boldsymbol{u} \in \mathbb{R}^3 }\mathop{\max}_{j \neq i} \mathop{\mathrm{ESDF}}(\boldsymbol{u}, \mathcal{S}_{ij}),     &\cap_{j \neq i} \mathcal{S}_{ij} = \varnothing,
   \end{aligned}
   \right.
\end{equation}
where $\mathop{\mathrm{ESDF}}(\boldsymbol{u}, \mathcal{S}_{ij})$ is the signed Euclidean distance function (ESDF) of $\boldsymbol{u}$ to the boundary of half-space $\mathcal{S}_{ij}$, with $\mathop{\mathrm{ESDF}}(\boldsymbol{u}, \mathcal{S}_{ij})$ defined to be negative if $\boldsymbol{u} \in \mathcal{S}_{ij}$.

\par If $\cap_{j\neq i} \mathcal{S}_{ij} \neq \varnothing$, $\boldsymbol{\pi}_i$ minimumly modifies $\boldsymbol{u}_{ri}$ to follow all ``safety protocols''.
Since the feasible region $\cap_{j\neq i} \mathcal{S}_{ij}$ is a polytope now, $\boldsymbol{\pi}_i$ is in the form of a quadratic programming (QP) and can be solved in real time onboard; 
If $\cap_{j\neq i} \mathcal{S}_{ij} = \varnothing$, it is then impossible to follow all ``safety protocols'' for all agent $j, j\neq i$. Therefore, $\boldsymbol{\pi}_i$ synthesises a control that minimizes the maximum violation of all ``safe protocols''.
Such an optimization can also be solved in real time onboard via a linear programming (LP).

\begin{thm}
   If $\cap_{j\neq i} \mathcal{S}_{ij} \neq \varnothing$ for all agent $i$, $\mathfrak{S}_0$ is forward invariant if $\boldsymbol{u}_i = \boldsymbol{\pi}_i(\boldsymbol{u}_{ri}, \boldsymbol{X})$ and $p_{ij} + p_{ji} \le 1, \forall i,j = 1, \dots, N, i\neq j$.
\end{thm}
\begin{pf}
   XXXX
\end{pf}

REMARK: meaning and plot, difference between CBF-QP

\section{Cooperating with high-level decisions}

\section{Numerical Experiments}

\section{Conclusion}

A conclusion section is not required. Although a conclusion may review
the main points of the paper, do not replicate the abstract as the
conclusion. A conclusion might elaborate on the importance of the work
or suggest applications and extensions.

\begin{ack}
Place acknowledgments here.
\end{ack}

\bibliography{ref}             % bib file to produce the bibliography
                                                     % with bibtex (preferred)

\end{document}
