%===============================================================================
% ifacconf.tex 2022-02-11 jpuente  
% 2022-11-11 jpuente change length of abstract
% Template for IFAC meeting papers
% Copyright (c) 2022 International Federation of Automatic Control
%===============================================================================
\documentclass{ifacconf}

\usepackage{graphicx}      % include this line if your document contains figures
\usepackage{natbib}        % required for bibliography
%===============================================================================
\begin{document}
\begin{frontmatter}

\title{Distributed Multi-satellite Collision Avoidance with Tunable Priority via Control Barrier Functions} 
% Title, preferably not more than 10 words.

\thanks[footnoteinfo]{Sponsor and financial support acknowledgment
goes here. Paper titles should be written in uppercase and lowercase
letters, not all uppercase.}

\author[First]{First A. Author} 
\author[Second]{Second B. Author, Jr.} 
\author[Third]{Third C. Author}

\address[First]{National Institute of Standards and Technology, 
   Boulder, CO 80305 USA (e-mail: author@ boulder.nist.gov).}
\address[Second]{Colorado State University, 
   Fort Collins, CO 80523 USA (e-mail: author@lamar. colostate.edu)}
\address[Third]{Electrical Engineering Department, 
   Seoul National University, Seoul, Korea, (e-mail: author@snu.ac.kr)}

\begin{abstract}                % Abstract of 50--100 words
   Miniaturization and clustering exacerbate collision risk of future satellites.
   In this paper, we propose a distributed framework for inter-satellite collision avoidance with tunable priority.
   We first deduce the collision-free condition of multi-satellite systems based on High Order Control Barrier Function techniques. 
   We then decouple such a condition for respective satellites, and further encode safety on the nominal controller in the form of distributed safety filter.
   By introducing priority parameter in the safety filter, the responsibility of evading collisions between satellite pairs becomes tunable, making it possible to modify the swarm behavior.
   Based on such a mechanism, we further showcase the safety filter's ability to cooperate with high level decisions: cooperating with optimization to approximate global optimal behavior and cooperating with Large Language Models to accommodate to tasks, respectively.
   Theoretical analysis have proved the safety guarantee of the safety filter and numerical Experiments validated the effectiveness of proposed method.
\end{abstract}

\begin{keyword}
Multi-satellite, Collision Avoidance, Control Barrier Function
\end{keyword}

\end{frontmatter}
%===============================================================================

\section{Introduction}
This document is a template for \LaTeXe. If you are reading a paper or
PDF version of this document, please download the electronic file
\texttt{ifacconf.tex}. You will also need the class file
\texttt{ifacconf.cls}. Both files are available on the IFAC web site.

Please stick to the format defined by the \texttt{ifacconf} class, and
do not change the margins or the general layout of the paper. It
is especially important that you do not put any running header/footer
or page number in the submitted paper.\footnote{
This is the default for the provided class file.}
Use \emph{italics} for emphasis; do not underline.

Page limits may vary from conference to conference. Please observe the 
page limits of the event for which your paper is intended. \cite{Abl:56}

\section{Preliminaries}

\section{Problem Formulation}

\section{Distributed Safety Filter Design}

\section{Cooperating with high-level decisions}

\section{Numerical Experiments}

\section{Conclusion}

A conclusion section is not required. Although a conclusion may review
the main points of the paper, do not replicate the abstract as the
conclusion. A conclusion might elaborate on the importance of the work
or suggest applications and extensions.

\begin{ack}
Place acknowledgments here.
\end{ack}

\bibliography{ifacconf}             % bib file to produce the bibliography
                                                     % with bibtex (preferred)

\end{document}
